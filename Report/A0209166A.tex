\documentclass[screen,nonacm]{acmart}

\usepackage{booktabs}
\usepackage{hyperref}
\usepackage{lipsum}
\usepackage{multicol}
\usepackage{tcolorbox}
\usepackage{xcolor}
\usepackage{subcaption}
\usepackage{adjustbox}
\usepackage{tikz}
\usetikzlibrary{positioning,shapes,arrows}

\usepackage{minted}
\setminted{
    autogobble=true,
    breaklines=true,
    fontsize=\footnotesize,
    xleftmargin=2em
}


\begin{document}

\title{
  Custom Acceleration with FPGAs
}
\subtitle{
  CS5222 -- Project 2
}

\author{Shen Jiamin (A0209166A)}
\email{shen_jiamin@u.nus.edu}

\begin{abstract}
  In this project, I'm going to port the lab to \textbf{PYNQ 2.7} and \textbf{Vivado/Vitis 2020.2}.
  The experiment is done on ASUS RS500-E8-PS4 V2, with operating system 
  Ubuntu 20.04.4 LTS (GNU/Linux 5.4.0-100-generic x86\_64).
\end{abstract}

\maketitle 

\section{Matrix Multiplication Pipeline Optimization in HLS}

%%%%%%%%%%%%%%%%%%%%%%%%%%%%%%%%%%%%%%%%%%%%%%%%%%%%%%%%%%%%%%%%%%%%%
\subsection{A. Understanding the baseline matrix multiply (background)}
%%%%%%%%%%%%%%%%%%%%%%%%%%%%%%%%%%%%%%%%%%%%%%%%%%%%%%%%%%%%%%%%%%%%%

The report generated by HLS (as in \autoref{rpt:float_00_orig}) shows that
some pipelining has already been done automatically by Vitis.
In order to prepare for the next part,
I disabled the pipelining.

\inputminted{diff}{program/float_nopipe.diff}

The new report is as \autoref{rpt:float_01_nopipe}.
It turns out that the overall performance is a little bit worse than documented.
This is because every iteration in L3 loop takes 11 cycles and thus
2816 cycles in total to perform a single inner product.


\begin{figure}
    {
        \subcaption{Performance Estimates}
        \inputminted[firstline=25,lastline=48]{text}{report/float_00_orig.rpt}
    }
    {
        \subcaption{Utilization Estimates}
        \inputminted[firstline=55,lastline=84]{text}{report/float_00_orig.rpt}
    }
    \caption{HLS Report in default condition}
    \label{rpt:float_00_orig}
\end{figure}


\begin{figure}
    {
        \subcaption{Performance Estimates}
        \inputminted[firstline=25,lastline=52]{text}{report/float_01_nopipe.rpt}
    }
    {
        \subcaption{Utilization Estimates}
        \inputminted[firstline=59,lastline=88]{text}{report/float_01_nopipe.rpt}
    }

    \caption{HLS Report with pipelining explicitly disabled}
    \label{rpt:float_01_nopipe}
\end{figure}

%%%%%%%%%%%%%%%%%%%%%%%%%%%%%%%%%%%%%%%%%%%%%%%%%%%%%%%%%%%%%%%%%%%%%
\subsection{B. Pipelining in HLS (8 marks)}
%%%%%%%%%%%%%%%%%%%%%%%%%%%%%%%%%%%%%%%%%%%%%%%%%%%%%%%%%%%%%%%%%%%%%

Report
\begin{enumerate}
    \item the design latency in cycles,
    \item the overall device utilization (as Total per Resource),
    \item the number of floating point adders and multipliers (you can find this information under the Instance section of the synthesis report) and
    \item the Initiation Interval of the loops you pipelined.
\end{enumerate}

%%%%%%%%%%%%%%%%%%%%%%%%%%%%%%%%%%%%%%%%%%%%%%%%%%%%%%%%%%%%%%%%%%%%%
\subsection{C. Increasing Pipeline Parallelism by Repartitioning Memories (8 marks)}
%%%%%%%%%%%%%%%%%%%%%%%%%%%%%%%%%%%%%%%%%%%%%%%%%%%%%%%%%%%%%%%%%%%%%

Report
\begin{enumerate}
    \item the design latency in cycles,
    \item the overall device utilization (as Total per Resource),
    \item the number of floating point adders and multipliers (you can find this information under the Instance section of the synthesis report) and
    \item the Initiation Interval of the loops you pipelined.
\end{enumerate}

%%%%%%%%%%%%%%%%%%%%%%%%%%%%%%%%%%%%%%%%%%%%%%%%%%%%%%%%%%%%%%%%%%%%%
\subsection{D. Amortizing Iteration Latency with Batching (8 marks)}
%%%%%%%%%%%%%%%%%%%%%%%%%%%%%%%%%%%%%%%%%%%%%%%%%%%%%%%%%%%%%%%%%%%%%

Report \begin{enumerate}
    \item the design latency in cycles, and
    \item the overall device utilization (as Total per Resource).
\end{enumerate}

%%%%%%%%%%%%%%%%%%%%%%%%%%%%%%%%%%%%%%%%%%%%%%%%%%%%%%%%%%%%%%%%%%%%%
\subsection{E. Extending Batch Size with Tiling (8 marks)}
%%%%%%%%%%%%%%%%%%%%%%%%%%%%%%%%%%%%%%%%%%%%%%%%%%%%%%%%%%%%%%%%%%%%%

Report \begin{enumerate}
    \item the design latency in cycles, and
    \item the overall device utilization (as Total per Resource).
\end{enumerate}

%%%%%%%%%%%%%%%%%%%%%%%%%%%%%%%%%%%%%%%%%%%%%%%%%%%%%%%%%%%%%%%%%%%%%
\subsection{F. Hardware compilation and FPGA testing on the PYNQ (8 marks)}
%%%%%%%%%%%%%%%%%%%%%%%%%%%%%%%%%%%%%%%%%%%%%%%%%%%%%%%%%%%%%%%%%%%%%

Report
\begin{enumerate}
    \item the measured speedup and
    \item measured classification accuracy.
\end{enumerate}


\section{Part 2: Fixed-Point Optimizations (30 marks)}

\begin{enumerate}
  \item the fixed-point validation accuracy reported by mnist.py after you've tweaked the SCALE factor.
  \item the design latency in cycles
  \item the overall device utilization (as Total per Resource).
  \item your measured system speedup over the fixed-point CPU implementation
  \item your measured classification accuracy on the 8k MNIST test sample
  \item how many multipliers are instantiated in your desing?
  \item report the initiation interval of the matrix multiplication loop that you pipelined
  \item given the number of multipliers in your design and input throughput via the AXI port, is the design bandwidth- or compute-limited?
\end{enumerate}

\section{Part 3: Open-ended design optimization (30 marks)}

\href{https://www.xilinx.com/support/documentation/sw_manuals/xilinx2020_2/ug1399-vitis-hls.pdf}{Vitis High-Level Synthesis User Guide}


% \bibliographystyle{ACM-Reference-Format}
% \bibliography{base}

\clearpage
\appendix

\end{document}
\endinput
