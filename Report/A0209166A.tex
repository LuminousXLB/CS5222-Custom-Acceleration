\documentclass[screen,nonacm]{acmart}

\usepackage{booktabs}
\usepackage{hyperref}
\usepackage{lipsum}
\usepackage{multicol}
\usepackage{tcolorbox}
\usepackage{xcolor}
\usepackage{adjustbox}
\usepackage{tikz}
\usetikzlibrary{positioning,shapes,arrows}

\usepackage{minted}
\setminted{
    autogobble=true,
    breaklines=true,
    fontsize=\small,
    xleftmargin=2em
}


\begin{document}

\title{
  Custom Acceleration with FPGAs
}
\subtitle{
  CS5222 -- Project 2
}

\author{Shen Jiamin (A0209166A)}
\email{shen_jiamin@u.nus.edu}


\begin{abstract}
  Your report in \texttt{.pdf} format with concise answers to the questions
  asked in Parts 1 and 2.
  In addition, the report should contain a short (couple paragraphs) description
  of your optimized implementation for Part 3 (what you changed about the
  hardware, or classifier, or both).
\end{abstract}

\maketitle

\section{Matrix Multiplication Pipeline Optimization in HLS}


\subsection{Understanding the baseline matrix multiply (background)}

\subsection{B. Pipelining in HLS (8 marks)}

Report
\begin{enumerate}
  \item the design latency in cycles,
  \item the overall device utilization (as Total per Resource),
  \item the number of floating point adders and multipliers (you can find this information under the Instance section of the synthesis report) and
  \item the Initiation Interval of the loops you pipelined.
\end{enumerate}

\subsection{C. Increasing Pipeline Parallelism by Repartitioning Memories (8 marks)}

Report
\begin{enumerate}
  \item the design latency in cycles,
  \item the overall device utilization (as Total per Resource),
  \item the number of floating point adders and multipliers (you can find this information under the Instance section of the synthesis report) and
  \item the Initiation Interval of the loops you pipelined.
\end{enumerate}

\subsection{D. Amortizing Iteration Latency with Batching (8 marks)}

Report \begin{enumerate}
  \item the design latency in cycles, and 
  \item the overall device utilization (as Total per Resource).
\end{enumerate}

\subsection{E. Extending Batch Size with Tiling (8 marks)}

Report \begin{enumerate}
  \item the design latency in cycles, and 
  \item the overall device utilization (as Total per Resource).
\end{enumerate}

\subsection{F. Hardware compilation and FPGA testing on the PYNQ (8 marks)}

Report 
\begin{enumerate}
  \item the measured speedup and 
  \item measured classification accuracy.
\end{enumerate}

\section{Part 2: Fixed-Point Optimizations (30 marks)}

\begin{enumerate}
  \item the fixed-point validation accuracy reported by mnist.py after you've tweaked the SCALE factor.
  \item the design latency in cycles
  \item the overall device utilization (as Total per Resource).
  \item your measured system speedup over the fixed-point CPU implementation
  \item your measured classification accuracy on the 8k MNIST test sample
  \item how many multipliers are instantiated in your desing?
  \item report the initiation interval of the matrix multiplication loop that you pipelined
  \item given the number of multipliers in your design and input throughput via the AXI port, is the design bandwidth- or compute-limited?
\end{enumerate}

\section{Part 3: Open-ended design optimization (30 marks)}


% \bibliographystyle{ACM-Reference-Format}
% \bibliography{base}

\clearpage
\appendix

\end{document}
\endinput
