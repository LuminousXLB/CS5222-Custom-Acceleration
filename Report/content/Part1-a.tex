%%%%%%%%%%%%%%%%%%%%%%%%%%%%%%%%%%%%%%%%%%%%%%%%%%%%%%%%%%%%%%%%%%%%%
\subsection{Understanding the baseline matrix multiply (background)}\label{sec:1a}
%%%%%%%%%%%%%%%%%%%%%%%%%%%%%%%%%%%%%%%%%%%%%%%%%%%%%%%%%%%%%%%%%%%%%

For Vitis 2020.2, the command used should be
\begin{minted}{console}
    $ vitis_hls -f hls.tcl
\end{minted}
The loop details report generated by HLS (as in \autoref{tab:float-loop-1a-baseline-autopipe}) shows that some pipelining has already been done automatically by Vitis HLS.
After inspecting the migration guide, I added two lines in the \texttt{hls.tcl}:
\begin{minted}{tcl}
    config_compile -pipeline_loops 0
    set_clock_uncertainty 12.5%
\end{minted}

The performance and utilization estimates of the two profiles, together with those for the following profiles, are reported in \autoref{tab:float-summary}.
The new loop details is as \autoref{tab:float-loop-1a-baseline-nopipe}.
It turns out that the overall performance is a little bit worse than documented.
This is because every iteration in L3 loop takes 11 cycles and thus
2816 cycles in total to perform a single inner product.

The overall latency of the baseline is 228022 cycles.
Since it only predicts on a batch of 8 inputs, the normalized latency is 28502.75 cycles.

\begin{table}[ht!]

    \caption{Loop details for baseline}

    \begin{subtable}{\textwidth}
        \caption{Baseline with automatic pipelining}
        \label{tab:float-loop-1a-baseline-autopipe}
        \centering
        \input{report/float-loop-1a-baseline-autopipe.tex}
    \end{subtable}

    \begin{subtable}{\textwidth}
        \caption{Baseline without automatic pipelining}
        \label{tab:float-loop-1a-baseline-nopipe}
        \centering
        \input{report/float-loop-1a-baseline-nopipe.tex}
    \end{subtable}

\end{table}

\begin{table}
    \caption{Performance and utilization estimates for \texttt{mmult\_float}}\label{tab:float-summary}
    {
\centering
\begin{tabularx}{\textwidth}{cl*{7}{C}}
    \toprule

    \multicolumn{2}{c}{\multirow{2}{*}{Profile}} &
    \multicolumn{2}{c}{Latency (cycles)}         &
    \multicolumn{2}{c}{Latency (ms)}             &
    \multicolumn{2}{c}{Interval (cycles)}        &
    \multicolumn{1}{c}{\multirow{2}{*}{\makecell*{Pipeline
                \\ Type}}}                                                                                                                \\

    \cmidrule(lr){3-4}
    \cmidrule(lr){5-6}
    \cmidrule(lr){7-8}
                                                 &
                                                 &
    \multicolumn{1}{c}{min}                      &
    \multicolumn{1}{c}{max}                      &
    \multicolumn{1}{c}{min}                      &
    \multicolumn{1}{c}{max}                      &
    \multicolumn{1}{c}{min}                      &
    \multicolumn{1}{c}{max}                      & \\
    \midrule
    \ref{sec:1a}                      & Baseline (AutoPipe) & 85160 & 85160 & 1.236 & 1.236 & 85161 & 85161 & none \\
\rowcolor{rowhlt}\ref{sec:1a}       & Baseline (NoPipe) & 228022 & 228022 & 2.280 & 2.280 & 228023 & 228023 & none \\
\ref{sec:1bL3}                          & L3 Pipelining & 85286 & 85286 & 1.238 & 1.238 & 85287 & 85287 & none \\
\ref{sec:1bL2}                     & L2 Pipelining (1WnR) & 7341 & 7341 & 0.073 & 0.073 & 7342 & 7342 & none \\
\rowcolor{rowhlt}\ref{sec:1bL2}     & L2 Pipelining (T2P) & 13885 & 13885 & 0.139 & 0.139 & 13886 & 13886 & none \\
\ref{sec:1bL1}                     & L1 Pipelining (1WnR) & 6193 & 6193 & 0.062 & 0.062 & 6194 & 6194 & none \\
\ref{sec:1bL1}                      & L1 Pipelining (T2P) & 5953 & 5953 & 0.060 & 0.060 & 5954 & 5954 & none \\
\ref{sec:1cDim}                     & Partition (\texttt{dim}=1, \texttt{factor}=2) & 16196 & 16196 & 0.162 & 0.162 & 16197 & 16197 & none \\
\rowcolor{rowhlt}\ref{sec:1cDim}    & Partition (\texttt{dim}=2, \texttt{factor}=2) & 11133 & 11133 & 0.111 & 0.111 & 11134 & 11134 & none \\
\ref{sec:1cFac}                     & Partition (\texttt{dim}=2, \texttt{factor}=4) & 8605 & 8605 & 0.086 & 0.086 & 8606 & 8606 & none \\
\ref{sec:1cFac}                     & Partition (\texttt{dim}=2, \texttt{factor}=8) & 9645 & 9645 & 0.096 & 0.096 & 9646 & 9646 & none \\
\rowcolor{rowhlt}\ref{sec:1cFac}   & Partition (\texttt{dim}=2, \texttt{factor}=16) & 9013 & 9013 & 0.090 & 0.090 & 9014 & 9014 & none \\
\ref{sec:1cFac}                    & Partition (\texttt{dim}=2, \texttt{factor}=32) & 8697 & 8697 & 0.087 & 0.087 & 8698 & 8698 & none \\
    \bottomrule
\end{tabularx}

\begin{tabularx}{\textwidth}{cl*{7}{C}}
    \toprule

    \multicolumn{2}{c}{\multirow{2}{*}{Profile}} &
    \multicolumn{5}{c}{Utilization Summary}      &
    \multicolumn{2}{c}{Instance}                                                                   \\

    \cmidrule(lr){3-7}
    \cmidrule(lr){8-9}
                                                 &
                                                 &
    \multicolumn{1}{c}{BRAM}                     &
    \multicolumn{1}{c}{DSP}                      &
    \multicolumn{1}{c}{FF}                       &
    \multicolumn{1}{c}{LUT}                      &
    \multicolumn{1}{c}{URAM}                     &
    \multicolumn{1}{c}{fadd}                     &
    \multicolumn{1}{c}{fmul}                                                                       \\

    \midrule
    \rowcolor{lime}                              & Available & 280 & 220 & 106400 & 53200 & 0 &  & \\
    \ref{sec:1a}                      & Baseline (AutoPipe) & 13 & 5 & 1151 & 2058 & 0 & 1 & 1 \\
\rowcolor{rowhlt}\ref{sec:1a}       & Baseline (NoPipe) & 14 & 5 & 817 & 1635 & 0 & 1 & 1 \\
\ref{sec:1bL3}                          & L3 Pipelining & 14 & 5 & 921 & 1713 & 0 & 1 & 1 \\
\ref{sec:1bL2}                     & L2 Pipelining (1WnR) & 182 & 80 & 38357 & 34359 & 0 & 16 & 16 \\
\rowcolor{rowhlt}\ref{sec:1bL2}     & L2 Pipelining (T2P) & 14 & 10 & 24774 & 22364 & 0 & 2 & 2 \\
\ref{sec:1bL1}                     & L1 Pipelining (1WnR) & 70 & 800 & 415044 & 243128 & 0 & 160 & 160 \\
\ref{sec:1bL1}                      & L1 Pipelining (T2P) & 14 & 100 & 313056 & 120190 & 0 & 20 & 20 \\
\ref{sec:1cDim}                     & Partition (\texttt{dim}=1, \texttt{factor}=2) & 14 & 10 & 32117 & 45376 & 0 & 2 & 2 \\
\rowcolor{rowhlt}\ref{sec:1cDim}    & Partition (\texttt{dim}=2, \texttt{factor}=2) & 14 & 20 & 27402 & 19150 & 0 & 4 & 4 \\
\ref{sec:1cFac}                     & Partition (\texttt{dim}=2, \texttt{factor}=4) & 18 & 40 & 31174 & 20654 & 0 & 8 & 8 \\
\ref{sec:1cFac}                     & Partition (\texttt{dim}=2, \texttt{factor}=8) & 34 & 80 & 37814 & 26771 & 0 & 16 & 16 \\
\rowcolor{rowhlt}\ref{sec:1cFac}   & Partition (\texttt{dim}=2, \texttt{factor}=16) & 66 & 160 & 52904 & 40646 & 0 & 32 & 32 \\
\ref{sec:1cFac}                    & Partition (\texttt{dim}=2, \texttt{factor}=32) & 130 & 320 & 72645 & 65969 & 0 & 64 & 64 \\
    \bottomrule
\end{tabularx}
}

\begin{enumerate}[nosep]
    \footnotesize
    \item ``\{L1, L2, L3\} Pipelining'' are based on Baseline (NoPipe).
    \item ``L2/Partition'' are based on L2 Pipelining.
\end{enumerate}
\end{table}
