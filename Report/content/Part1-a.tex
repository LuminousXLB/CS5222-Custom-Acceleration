%%%%%%%%%%%%%%%%%%%%%%%%%%%%%%%%%%%%%%%%%%%%%%%%%%%%%%%%%%%%%%%%%%%%%
\subsection{Understanding the baseline matrix multiply (background)}\label{sec:1a}
%%%%%%%%%%%%%%%%%%%%%%%%%%%%%%%%%%%%%%%%%%%%%%%%%%%%%%%%%%%%%%%%%%%%%

For Vitis 2020.2, the command used should be
\begin{minted}{console}
    $ vitis_hls -f hls.tcl
\end{minted}
The report generated by HLS (as in \autoref{tab:float-loop-1a-baseline-autopipe}) shows that some pipelining has already been done automatically by Vitis HLS.
After inspecting the migration guide, I added two lines in the \texttt{hls.tcl}:
\begin{minted}{tcl}
    config_compile -pipeline_loops 0
    set_clock_uncertainty 12.5%
\end{minted}

The performance and utilization estimates of the two profiles, together with those for the following profiles, are reported in \autoref{tab:float-summary}.
The new loop details is as \autoref{tab:float-loop-1a-baseline-nopipe}.
It turns out that the overall performance is a little bit worse than documented.
This is because every iteration in L3 loop takes 11 cycles and thus
2816 cycles in total to perform a single inner product.

\begin{table}
    \caption{Loop details for baseline with automatic pipelining}
    \label{tab:float-loop-1a-baseline-autopipe}
    \centering
    \begin{tabularx}{\textwidth}{ p{4cm} *{7}{C}}
    \toprule
    \multicolumn{1}{c}{\multirow{2}{*}{Loop Name}} &
    \multicolumn{2}{c}{Latency (cycles)}           &
    \multicolumn{1}{c}{\multirow{2}{*}{\makecell*{Iteration \\
    Latency}}}                                     &
    \multicolumn{2}{c}{Initiation Interval}        &
    \multicolumn{1}{c}{\multirow{2}{*}{\makecell*{Trip      \\
    Count}}}                                       &
    \multicolumn{1}{c}{\multirow{2}{*}{Pipelined}}          \\

    \cmidrule(lr){2-3}
    \cmidrule(lr){5-6}

                                                   &
    \multicolumn{1}{c}{min}                        &
    \multicolumn{1}{c}{max}                        &
                                                   &
    \multicolumn{1}{c}{achieved}                   &
    \multicolumn{1}{c}{target}                     &        \\
    \midrule
    \texttt{- LOAD\_OFF\_1} & 5 & 5 & 1 & 1 & 1 & 5 & yes \\
\texttt{- LOAD\_W\_1\_LOAD\_W\_2} & 1280 & 1280 & 1 & 1 & 1 & 1280 & yes \\
\texttt{- LOAD\_I\_1\_LOAD\_I\_2} & 1024 & 1024 & 1 & 1 & 1 & 1024 & yes \\
\texttt{- L1\_L2} & 82800 & 82800 & 1035 & - & - & 80 & no \\
\texttt{ + L3} & 1031 & 1031 & 12 & 4 & 1 & 256 & yes \\
\texttt{- STORE\_O\_1\_STORE\_O\_2} & 42 & 42 & 4 & 1 & 1 & 40 & yes \\
    \bottomrule
\end{tabularx}

\end{table}

\begin{table}
    \caption{Loop details for baseline without automatic pipelining}
    \label{tab:float-loop-1a-baseline-nopipe}
    \centering
    \begin{tabularx}{\textwidth}{ p{4cm} *{7}{C}}
    \toprule
    \multicolumn{1}{c}{\multirow{2}{*}{Loop Name}} &
    \multicolumn{2}{c}{Latency (cycles)}           &
    \multicolumn{1}{c}{\multirow{2}{*}{\makecell*{Iteration \\
    Latency}}}                                     &
    \multicolumn{2}{c}{Initiation Interval}        &
    \multicolumn{1}{c}{\multirow{2}{*}{\makecell*{Trip      \\
    Count}}}                                       &
    \multicolumn{1}{c}{\multirow{2}{*}{Pipelined}}          \\

    \cmidrule(lr){2-3}
    \cmidrule(lr){5-6}

                                                   &
    \multicolumn{1}{c}{min}                        &
    \multicolumn{1}{c}{max}                        &
                                                   &
    \multicolumn{1}{c}{achieved}                   &
    \multicolumn{1}{c}{target}                     &        \\
    \midrule
    \texttt{- LOAD\_OFF\_1} & 5 & 5 & 1 & - & - & 5 & no \\
\texttt{- LOAD\_W\_1} & 1300 & 1300 & 130 & - & - & 10 & no \\
\texttt{ + LOAD\_W\_2} & 128 & 128 & 1 & - & - & 128 & no \\
\texttt{- LOAD\_I\_1} & 1040 & 1040 & 130 & - & - & 8 & no \\
\texttt{ + LOAD\_I\_2} & 128 & 128 & 1 & - & - & 128 & no \\
\texttt{- L1} & 225536 & 225536 & 28192 & - & - & 8 & no \\
\texttt{ + L2} & 28190 & 28190 & 2819 & - & - & 10 & no \\
\texttt{  ++ L3} & 2816 & 2816 & 11 & - & - & 256 & no \\
\texttt{- STORE\_O\_1} & 136 & 136 & 17 & - & - & 8 & no \\
\texttt{ + STORE\_O\_2} & 15 & 15 & 3 & - & - & 5 & no \\
    \bottomrule
\end{tabularx}

\end{table}


\begin{table}
    \caption{Performance and utilization estimates for \texttt{mmult\_float}}\label{tab:float-summary}
    {
\small
\centering
\begin{tabularx}{\textwidth}{cl*{6}{R}C}
    \toprule

    \multicolumn{2}{c}{\multirow{2}{*}{Profile}} &
    \multicolumn{2}{c}{Latency (cycles)}         &
    \multicolumn{2}{c}{Latency (ms)}             &
    \multicolumn{2}{c}{Interval (cycles)}        &
    \multicolumn{1}{c}{\multirow{2}{*}{\makecell*{Pipeline
                \\ Type}}}                                                                                                                \\

    \cmidrule(lr){3-4}
    \cmidrule(lr){5-6}
    \cmidrule(lr){7-8}
                                                 &
                                                 &
    \multicolumn{1}{c}{min}                      &
    \multicolumn{1}{c}{max}                      &
    \multicolumn{1}{c}{min}                      &
    \multicolumn{1}{c}{max}                      &
    \multicolumn{1}{c}{min}                      &
    \multicolumn{1}{c}{max}                      & \\
    \midrule
    \ref{sec:1a}                      & Baseline (AutoPipe) & 85160 & 85160 & 1.236 & 1.236 & 85161 & 85161 & none \\
\rowcolor{rowhlt}\ref{sec:1a}       & Baseline (NoPipe) & 228022 & 228022 & 2.280 & 2.280 & 228023 & 228023 & none \\
\ref{sec:1bL3}                          & L3 Pipelining & 85286 & 85286 & 1.238 & 1.238 & 85287 & 85287 & none \\
\ref{sec:1bL2}                     & L2 Pipelining (1WnR) & 7341 & 7341 & 0.073 & 0.073 & 7342 & 7342 & none \\
\ref{sec:1bL2}                     & L2 Pipelining (T2P) & 13885 & 13885 & 0.139 & 0.139 & 13886 & 13886 & none \\
\ref{sec:1bL1}                     & L1 Pipelining (1WnR) & 6193 & 6193 & 0.062 & 0.062 & 6194 & 6194 & none \\
\ref{sec:1bL1}                      & L1 Pipelining (T2P) & 5953 & 5953 & 0.060 & 0.060 & 5954 & 5954 & none \\
\rowcolor{rowhlt}\ref{sec:1c}  & Baseline (L2, AutoPipe, T2P) & 13759 & 13759 & 0.138 & 0.138 & 13760 & 13760 & none \\
\ref{sec:1cDim}                     & Partition (\texttt{dim}=1, \texttt{factor}=2) & 13772 & 13772 & 0.138 & 0.138 & 13773 & 13773 & none \\
\rowcolor{rowhlt}\ref{sec:1cDim}    & Partition (\texttt{dim}=2, \texttt{factor}=2) & 8703 & 8703 & 0.087 & 0.087 & 8704 & 8704 & none \\
\ref{sec:1cFac}                     & Partition (\texttt{dim}=2, \texttt{factor}=4) & 6175 & 6175 & 0.062 & 0.062 & 6176 & 6176 & none \\
\ref{sec:1cFac}                     & Partition (\texttt{dim}=2, \texttt{factor}=8) & 4911 & 4911 & 0.049 & 0.049 & 4912 & 4912 & none \\
\rowcolor{rowhlt}\ref{sec:1cFac}   & Partition (\texttt{dim}=2, \texttt{factor}=16) & 4279 & 4279 & 0.043 & 0.043 & 4280 & 4280 & none \\
\ref{sec:1cFac}                    & Partition (\texttt{dim}=2, \texttt{factor}=32) & 3963 & 3963 & 0.040 & 0.040 & 3964 & 3964 & none \\
    \bottomrule
\end{tabularx}

\begin{tabularx}{\textwidth}{cl*{5}{c}*{2}{C}}
    \toprule

    \multicolumn{2}{c}{\multirow{2}{*}{Profile}} &
    \multicolumn{5}{c}{Utilization Summary}      &
    \multicolumn{2}{c}{Instance}                   \\

    \cmidrule(lr){3-7}
    \cmidrule(lr){8-9}
                                                 &
                                                 &
    \multicolumn{1}{c}{BRAM\_18K}                &
    \multicolumn{1}{c}{DSP}                      &
    \multicolumn{1}{c}{FF}                       &
    \multicolumn{1}{c}{LUT}                      &
    \multicolumn{1}{c}{URAM}                     &
    \multicolumn{1}{c}{fadd}                     &
    \multicolumn{1}{c}{fmul}                       \\

    \midrule
    \ref{sec:1a}                      & Baseline (AutoPipe) & 13 (4\%) & 5 (2\%) & 1050 (\textasciitilde 0\%) & 2000 (3\%) & 0 (0\%) & 1 & 1 \\
\rowcolor{rowhlt}\ref{sec:1a}       & Baseline (NoPipe) & 14 (5\%) & 5 (2\%) & 817 (\textasciitilde 0\%) & 1635 (3\%) & 0 (0\%) & 1 & 1 \\
\ref{sec:1bL3}                          & L3 Pipelining & 14 (5\%) & 5 (2\%) & 921 (\textasciitilde 0\%) & 1713 (3\%) & 0 (0\%) & 1 & 1 \\
\ref{sec:1bL2}                     & L2 Pipelining (1WnR) & 182 (65\%) & 80 (36\%) & 38357 (36\%) & 34359 (64\%) & 0 (0\%) & 16 & 16 \\
\ref{sec:1bL2}                     & L2 Pipelining (T2P) & 16 (5\%) & 10 (4\%) & 24710 (23\%) & 22359 (42\%) & 0 (0\%) & 2 & 2 \\
\ref{sec:1bL1}                     & L1 Pipelining (1WnR) & 70 (25\%) & 800 (363\%) & 415044 (390\%) & 243128 (457\%) & 0 (0\%) & 160 & 160 \\
\ref{sec:1bL1}                      & L1 Pipelining (T2P) & 16 (5\%) & 100 (45\%) & 312992 (294\%) & 120185 (225\%) & 0 (0\%) & 20 & 20 \\
\rowcolor{rowhlt}\ref{sec:1c}  & Baseline (L2, AutoPipe, T2P) & 16 (5\%) & 10 (4\%) & 24776 (23\%) & 22615 (42\%) & 0 (0\%) & 2 & 2 \\
\ref{sec:1cDim}                     & Partition (\texttt{dim}=1, \texttt{factor}=2) & 16 (5\%) & 10 (4\%) & 32491 (30\%) & 45788 (86\%) & 0 (0\%) & 2 & 2 \\
\rowcolor{rowhlt}\ref{sec:1cDim}    & Partition (\texttt{dim}=2, \texttt{factor}=2) & 16 (5\%) & 20 (9\%) & 27306 (25\%) & 19326 (36\%) & 0 (0\%) & 4 & 4 \\
\ref{sec:1cFac}                     & Partition (\texttt{dim}=2, \texttt{factor}=4) & 20 (7\%) & 40 (18\%) & 31022 (29\%) & 20786 (39\%) & 0 (0\%) & 8 & 8 \\
\ref{sec:1cFac}                     & Partition (\texttt{dim}=2, \texttt{factor}=8) & 36 (12\%) & 80 (36\%) & 37606 (35\%) & 26782 (50\%) & 0 (0\%) & 16 & 16 \\
\rowcolor{rowhlt}\ref{sec:1cFac}   & Partition (\texttt{dim}=2, \texttt{factor}=16) & 68 (24\%) & 160 (72\%) & 52606 (49\%) & 40402 (75\%) & 0 (0\%) & 32 & 32 \\
\ref{sec:1cFac}                    & Partition (\texttt{dim}=2, \texttt{factor}=32) & 132 (47\%) & 320 (145\%) & 72201 (67\%) & 65294 (122\%) & 0 (0\%) & 64 & 64 \\
    \bottomrule
\end{tabularx}
}

\begin{enumerate}[nosep]
    \footnotesize
    \item ``\{L1, L2, L3\} Pipelining'' are based on Baseline (NoPipe).
    \item ``L2/Partition'' are based on L2 Pipelining.
\end{enumerate}
\end{table}
