%%%%%%%%%%%%%%%%%%%%%%%%%%%%%%%%%%%%%%%%%%%%%%%%%%%%%%%%%%%%%%%%%%%%%
\subsection{Understanding the baseline matrix multiply (background)}\label{sec:1a}
%%%%%%%%%%%%%%%%%%%%%%%%%%%%%%%%%%%%%%%%%%%%%%%%%%%%%%%%%%%%%%%%%%%%%

For Vitis 2020.2, the command used should be
\begin{minted}{console}
    $ vitis_hls -f hls.tcl
\end{minted}
The report generated by HLS (as in \autoref{tab:float-loop-00-baseline-autopipe}) shows that
some pipelining has already been done automatically by Vitis HLS.
In order to prepare baseline for the next part, I disabled the pipelining.

\inputminted{diff}{program/float_nopipe.diff}

The new report is as \autoref{tab:float-loop-01-baseline-nopipe}.
It turns out that the overall performance is a little bit worse than documented.
This is because every iteration in L3 loop takes 11 cycles and thus
2816 cycles in total to perform a single inner product.

\begin{table}
    \caption{Loop details for baseline with automatic pipelining}
    \label{tab:float-loop-00-baseline-autopipe}
    \centering
    \begin{tabularx}{\textwidth}{ p{4cm} *{7}{C}}
    \toprule
    \multicolumn{1}{c}{\multirow{2}{*}{Loop Name}} &
    \multicolumn{2}{c}{Latency (cycles)}           &
    \multicolumn{1}{c}{\multirow{2}{*}{\makecell*{Iteration \\
    Latency}}}                                     &
    \multicolumn{2}{c}{Initiation Interval}        &
    \multicolumn{1}{c}{\multirow{2}{*}{\makecell*{Trip      \\
    Count}}}                                       &
    \multicolumn{1}{c}{\multirow{2}{*}{Pipelined}}          \\

    \cmidrule(lr){2-3}
    \cmidrule(lr){5-6}

                                                   &
    \multicolumn{1}{c}{min}                        &
    \multicolumn{1}{c}{max}                        &
                                                   &
    \multicolumn{1}{c}{achieved}                   &
    \multicolumn{1}{c}{target}                     &        \\
    \midrule
    \texttt{- LOAD\_OFF\_1} & 5 & 5 & 1 & 1 & 1 & 5 & yes \\
\texttt{- LOAD\_W\_1\_LOAD\_W\_2} & 1280 & 1280 & 2 & 1 & 1 & 1280 & yes \\
\texttt{- LOAD\_I\_1\_LOAD\_I\_2} & 1024 & 1024 & 2 & 1 & 1 & 1024 & yes \\
\texttt{- L1\_L2} & 82800 & 82800 & 1035 & - & - & 80 & no \\
\texttt{ + L3} & 1031 & 1031 & 12 & 4 & 1 & 256 & yes \\
\texttt{- STORE\_O\_1\_STORE\_O\_2} & 42 & 42 & 4 & 1 & 1 & 40 & yes \\
    \bottomrule
\end{tabularx}

\end{table}

\begin{table}
    \caption{Loop details for baseline without automatic pipelining}
    \label{tab:float-loop-01-baseline-nopipe}
    \centering
    \begin{tabularx}{\textwidth}{ p{4cm} *{7}{C}}
    \toprule
    \multicolumn{1}{c}{\multirow{2}{*}{Loop Name}} &
    \multicolumn{2}{c}{Latency (cycles)}           &
    \multicolumn{1}{c}{\multirow{2}{*}{\makecell*{Iteration \\
    Latency}}}                                     &
    \multicolumn{2}{c}{Initiation Interval}        &
    \multicolumn{1}{c}{\multirow{2}{*}{\makecell*{Trip      \\
    Count}}}                                       &
    \multicolumn{1}{c}{\multirow{2}{*}{Pipelined}}          \\

    \cmidrule(lr){2-3}
    \cmidrule(lr){5-6}

                                                   &
    \multicolumn{1}{c}{min}                        &
    \multicolumn{1}{c}{max}                        &
                                                   &
    \multicolumn{1}{c}{achieved}                   &
    \multicolumn{1}{c}{target}                     &        \\
    \midrule
    \texttt{- LOAD\_OFF\_1} & 5 & 5 & 1 & - & - & 5 & no \\
\texttt{- LOAD\_W\_1} & 1300 & 1300 & 130 & - & - & 10 & no \\
\texttt{ + LOAD\_W\_2} & 128 & 128 & 1 & - & - & 128 & no \\
\texttt{- LOAD\_I\_1} & 1040 & 1040 & 130 & - & - & 8 & no \\
\texttt{ + LOAD\_I\_2} & 128 & 128 & 1 & - & - & 128 & no \\
\texttt{- L1} & 225536 & 225536 & 28192 & - & - & 8 & no \\
\texttt{ + L2} & 28190 & 28190 & 2819 & - & - & 10 & no \\
\texttt{  ++ L3} & 2816 & 2816 & 11 & - & - & 256 & no \\
\texttt{- STORE\_O\_1} & 136 & 136 & 17 & - & - & 8 & no \\
\texttt{ + STORE\_O\_2} & 15 & 15 & 3 & - & - & 5 & no \\
    \bottomrule
\end{tabularx}

\end{table}

The performance and utilization estimates of the two profiles, together with those for the following profiles, are reported in \autoref{tab:float-summary}.
