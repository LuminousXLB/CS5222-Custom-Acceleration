%%%%%%%%%%%%%%%%%%%%%%%%%%%%%%%%%%%%%%%%%%%%%%%%%%%%%%%%%%%%%%%%%%%%%
\subsection{Hardware compilation and FPGA testing on the PYNQ (8 marks)}\label{sec:1f}
%%%%%%%%%%%%%%%%%%%%%%%%%%%%%%%%%%%%%%%%%%%%%%%%%%%%%%%%%%%%%%%%%%%%%

In this section, we met some problems.

The first problem that I met is when exporting the IP, which has been sovled in: \href{https://support.xilinx.com/s/question/0D52E00006uxy49SAA/vivado-fails-to-export-ips-with-the-error-message-bad-lexical-cast-source-type-value-could-not-be-interpreted-as-target}{Vivado fails to export IPs with the error message "Bad lexical cast: source type value could not be interpreted as target"}.

The second problem occurs when validating the system design, which reports a critical warning as \autoref{fig:tlast}.
As this signla shall be provided by Vitis HLS when implementing the AXI Stream interface, I changed the signature of the top function from C array to \texttt{hls::stream}.
That fixes the problem.

\begin{figure}
    \begin{minted}{text}
        [xilinx.com:ip:axi_dma:7.1-9] /axi_dma_1
        ###########################################################
        Interface connected to S_AXIS_S2MM does not have TLAST port
        ###########################################################
    \end{minted}
    \caption{Critical Warning from validating the design}
    \label{fig:tlast}
\end{figure}

After building, move the bitstream files to the board.

\begin{minted}{console}
    $ scp `find -name "*.bit"` xilinx@<pynq-ip-addr>:~/classifier.bit
    $ scp `find -name "*.hwh"` xilinx@<pynq-ip-addr>:~/classifier.hwh
\end{minted}

According to the different versions of PYNQ, I have to modify the notebook as well.
The new version is attached in the submission.

The measured misclassification rate is 13.04\% for both FPGA and CPU, which indicates that our implementation is correct.
The running time on FPGA is around 16.16ms, while 78.04ms on CPU.
That is, speedup on FPGA is 4.83x, a little bit smaller than required.
This may due to the optimization between different versions of python and numpy.
